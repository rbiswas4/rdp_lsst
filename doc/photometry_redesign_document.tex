\documentclass{article}[12pt]
\begin{document}
\section{Frequently Desired Operations}

\begin{itemize}
    \item Find fluxes in a list of transmission bands and times of a large number of SEDs. The wavelengths of SEDS may be at different grids. The use case is to find the fluxes/ mags of every object of a certain class (galaxy/AGN/SN) on a focal plane. Since these might be at different redshifts, the wavelength grid could be different), unless we store that at the top of the atmosphere on the same wavelength grid.
        So we are looking at 
\begin{verbatim}
# Fix all SEDs to the bandpass wavelength grid
bandpassdict.flux([sed0, sed1, sed2, sed3], bandpassindices=['u', 'g', 'r', 'i'])
\end{verbatim}
where the length of bandpassindices matches the length of the bandpass string
    \item Transients: The focal plane will have a large number of transients 
\end{itemize}

\section{Gloassary: Definitions, Conventions and Equations}

\subsection{Transmission Functions} $T(\lambda)$ 

\subsection{Specific Flux} 

\subsection{Flux} As it is measured by our CCDs, the important quantity is 
number of photons per unit area per unit time. The units are thus
$cm^{-2} s^{-1}$. A second unit of interest is the ratio of this quantity for our source to that of a standard source as defined by the Magnitude System.
\begin{equation}
\rm{Flux} = \int \frac{dE(\lambda)}{d\lambda} T(\lambda)
    \frac{\lambda d\lambda}{hc}
\end{equation}
We can also write this as 
\begin{equation}
    \rm{Flux} = \int \frac{dE(\lambda)}{d\nu} T(\lambda) \frac{d\nu}{h\nu}
\end{equation}
\subsection{Magnitude System}

\subsection{Magnitudes}

\section{Data Structures and PseudoCode}

\subsection{Transmission Functions} 
\begin{itemize}
    \item The main class of objects should be a Bandpass class, and a Bandpassdict class. It is expected that the bandpassdict class will encapsulate all the bands of a survey and will be the most commonly used object. The user should try to obtain the appropriate bandpassdict and use it.
    \item use a registry to be able to refer to the bandpasses and bandpassdictsthrough strings (connected to next point)
    \item Use builtins for well known transmission functions so that instantiation is unnecessary, along with the ability to add transmission functions through an instantiation of bandpass/bandpassdict objects
    \item Not really throught about: How to extend bandpassdict to heterogeneous transmission function models?
\end{itemize}
\subsubsection{Attributes of BandPass}
\end{document}
